\section{Forward Measure}

\subsection{Intuitive Definition}

To start with, recall in the HJM framework we have the discount value of a bond price is a martingale in risk neutral measure. This is true
because any tradeable assets must be a martingale in risk neutral measure otherwise you have arbitrage oppurtunity

\begin{equation}
  d(D(t)B(t, T)) = -\sigma^{\ast}D(t)B(t, T)d\widetilde{W}(t)
\end{equation}

Then recall the definition of the fundamental theorem that there exists a measure between two tradeable assets that those two can form a martingale.

If we further assume the measure is at time T, which is the same expiration time of the zero-coupon bonds, then we have:

\begin{equation}
  \frac{V(t)}{B(0, T)} = E^{T}\frac{V(T)}{B(T, T)} = E^{T}V(T)
\end{equation}

\begin{equation}
  V(t) = B(0, T)E^{T}V(T)
\end{equation}

The idea behind this is that, we cannot find a way using bond to hedge our payoff then get money at any condition. Recall the similar risk neutral
pricing formula for any contract. But first, think of risk neutral measure as a measure that you can find a martingale such that it prevents you from being using
bank account to make arbitrage

\begin{equation}
  V(t) = E^{B}(V(T)D(T))
\end{equation}


Since T, B measures are both defined on same event space. The we can naturally get the transformation of the probability on same event to be

\begin{equation}
  \widetilde{P}^{T}(A) = \frac{1}{B(0, T)} \int_A D(T) d\widetilde{P}^B(A)
\end{equation}


\subsection{Link To Risk Neutral Measure}
\subsubsection{Non-Model Based}

We often hear someone say {\color{red}\textit{forward price is martingale under T measure}} ({\color{blue}\textit{1 over forward price is martingale under S measure})}.  Firstly, without proof, this statement can be think of as:
\begin{itemize}
\item First forward price is the number of bond you hold to hedge the price at time T. Aka the ratio between the expected payoff of contract vs the bond
\item Under T measure, this value should equal to $\frac{V(t)}{B(t, T)}$ at time t
\item In fact, the ratio value mentioned in first item should equal to $\frac{V(0)}{B(0, T)} $ at time 0
\item According to the {\color{red}Tower Theorem, this second and third items are chained up to force the evolution of $f$ from time 0 to time t, the f is a martingale}
\end{itemize}

More intuitively, you can see using T measure just include/offset the factor of $B(0, T)$ and $D(T)$ in the risk neutral measure.

\begin{equation}
  f(0) = \frac{E^{B} V(T) D(T)}{E^BD(T)} = B(0, T)E^{B} V(T) D(T) =  E^{T}f(T)
\end{equation}


\subsubsection{Change of Numeraire}
Thinking intuitively, T measure is just some drift transformation of normal distribution. Under risk-neutral measure, assume we have two process $S(t)$ and $N(t)$ (both are tradable assets).

\begin{equation}
  D(t)S(t) = D(0) S(0) \exp(\int_{0}^{t} \sigma(t) d\widetilde{W}(u) - 0.5\int_{0} ^ {t} \norm{\sigma(t) ^ 2} du)
\end{equation}
\begin{equation}
  D(t)N(t) = D(0) N(0) \exp(\int_{0}^{t} \mu(t) d\widetilde{W}(u) - 0.5\int_{0} ^ {t} \norm{\mu(t) ^ 2} du)
\end{equation}

So therefore under this risk neutral pricing measure, we can have
\begin{equation}
  \begin{aligned}
  \frac{S(t)}{N(t)} &= \frac{S(0)}{N(0)} \exp(\int_{0}^{t} (\sigma(t) - \mu(t)) d\widetilde{W}(u) - 0.5\int_{0} ^ {t} (\norm{\sigma(t) ^ 2} - \norm{\mu(t) ^ 2}) du) \\
                    &= \frac{S(0)}{N(0)} \exp(\int_{0}^{t} (\sigma(t) - \mu(t)) d\widetilde{W}(u) - 0.5\int_{0} ^ {t} (\sigma(t) - \mu(t)) ^ 2 du) - \int_0^t\sum_{1} ^ {d} \mu_{i}(t)(\sigma_{i}(t) - \mu_{i}(t)) du
  \end{aligned}
\end{equation}

So we can find a measure

\begin{equation}
  \begin{aligned}
    d\widetilde{W}^N(u) &= \begin{bmatrix}
            d\widetilde{W}_{1}  - \mu_1(u)\\
            d\widetilde{W}_{2}  - \mu_2(u)\\
            \vdots \\
            d\widetilde{W}_{d}  - \mu_d(u)
          \end{bmatrix}
    \end{aligned}
  \label{measure_change}
\end{equation}

This equation is telling you that {\color{red}under measure N, the normal distribution in risk neutral have positive drift $\mu(u)$}

\begin{exmp}
Under single stock measure, the log normal stock actually have $\sigma ^ 2$ drift
\begin{equation}
  \begin{aligned}
  d(\frac{\log(S)}{S}) = (r - \frac{1}{2} \sigma^2)dt + \sigma \cdot \sigma dt + \sigma d\widetilde{W}^S(t)
  \end{aligned}
\end{equation}
\end{exmp}

\begin{exmp}
Quanto Option, suppose XAU-EUR is $S_1(t)$ and USD-EUR $S_2(t)$. Then we have
\begin{equation}
  \begin{aligned}
  \frac{d(S_1)}{S_1} = r_f dt + \sigma_1 d\widetilde{W_1}^f(t)
  \end{aligned}
\end{equation}
\begin{equation}
  \begin{aligned}
  \frac{d(S_2)}{S_2} = (r_f - r_d) dt + \sigma_2 d\widetilde{W_2}^f(t)
  \end{aligned}
\end{equation}

The inverse of exchange rate process under EUR is

\begin{equation}
d(\frac{1}{S_2}) = \frac{1}{S_2} ( (r_d - r_f + \sigma_2^2) dt - \sigma_2 d\widetilde{W_2}^f(t) )
\end{equation}

Under EUR risk neutral measure, the XAU by USD have (using $dXY = XdY + Xdx + dXdY$)

\begin{equation}
\frac{dS_3}{S_3} = (r_d + \sigma_2^2 - \rho\sigma_1\sigma_2) dt  + \sigma_1 d\widetilde{W_1}^f(t) - \sigma_2 d\widetilde{W_2}^f(t)
\end{equation}

when we change measure EUR to measure USD then we have

\begin{equation}
\begin{aligned}
\frac{dS_3}{S_3} &= (r_d + \sigma_2^2 - \rho\sigma_1\sigma_2) dt  + (\sigma_1 - \rho\sigma_2)(d\widetilde{W_1}^d(t) + \rho \sigma_2 dt) \\
                 &- \sigma_2 \sqrt{1 - \rho ^ 2} (d\widetilde{W_2}^d(t) + \sqrt{1 - \rho ^ 2} \sigma_2 dt\\
                 &= r_d dt + \sqrt{\sigma_1^2 + \sigma_2^2 - 2\rho \sigma_1 \sigma_2} \cdot d\widetilde{W_3}^d(t)
\end{aligned}
\end{equation}

The last equation shows that {\color{blue}after exchange rate consideration, the XAU in USD is still a martingale under USD measure.}

\end{exmp}

\subsection{Normal Based T Measure}
Based on the analysis above. Since discount zero coupon bond is a martingale under risk neutral measure.
\begin{equation}
\begin{aligned}
d(D(t)B(t, T)) = D(t)B(t, T) - \sigma^\ast(t) d\widetilde{W}(t)
\end{aligned}
\end{equation}
Use equation \ref{measure_change} can effciently get
\begin{equation}
\begin{aligned}
d\widetilde{W}^T(t) - \sigma^\ast(t) = d\widetilde{W}(t)
\end{aligned}
\end{equation}

\subsection{Use T Measure To Price Option}
Even if interest rate is random, we can still get BS formula in a very general way.

\begin{equation}
\begin{aligned}
\widetilde{E}(D(T)(S(T) - K))^+ &= \widetilde{E}(D(T)S(T) \mathbbm{1} [ S(T) > K ]) - K\widetilde{E}(D(T)\mathbbm{1} [ S(T) > K ])
\end{aligned}
\end{equation}

The second part is easy to show. {\color{red} The $\sigma$ here is the vol of forward price, which is comprised of both vol of spot and vol of zero bond(interest rate)}
\begin{equation}
\begin{aligned}
 K\widetilde{E}(D(T)\mathbbm{1} [ S(T) > K ]) &= KB(0, T) \widetilde{E} ^ T (F(T, T) > K) \\
                                              &= KB(0, T) P^T(\exp(-\frac{1}{2}\sigma^2t + \sigma\widetilde{W}^T(T)) > \frac{K}{F(0, T)}) \\
                                              &= KB(0, T) N(d_2)
\end{aligned}
\end{equation}

The first part is quite similar
\begin{equation}
\begin{aligned}
\widetilde{E}(D(T)S(T) \mathbbm{1} [ S(T) > K ]) &= S(0) \widetilde{E} ^ S (F(T, T) > K) \\
\end{aligned}
\end{equation}

The tricky part is in fact, under the definition of S measure, the inverse of F is a martingale with vol $-\sigma$
\begin{equation}
\begin{aligned}
\frac{B(0, T)}{S(0)} = \frac{1}{F(0, T)} = E^S(\frac{B(t, T)}{S(t)}) = E^S(\frac{1}{F(t, T)}) \\
\end{aligned}
\end{equation}

The using the similar technique as for second part proof we can get the expression for first term.
\begin{equation}
\begin{aligned}
S(0) N(d_1)
\end{aligned}
\end{equation}
