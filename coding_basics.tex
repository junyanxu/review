\section{Coding Basics}
\subsection{C++}
\subsubsection{Template}

Key things to know for template:
\begin{enumerate}
\item as long as the type is specified clearly, template class will be initiated at compile type
\item you can write a factorial calculation at using template at complied time. also {\color{blue}constexpr} will achieve this function as well
\item we can achieve virtual function by using template
\end{enumerate}

\begin{lstlisting}
#include<iostream>
#include<string>
using namespace std;

template <typename T> void show(const T& data){
  if(typeid(data) == typeid(string("abc")))
    cout << "This is a string" << endl;
  if(typeid(data) == typeid(1))
    cout << "This is a int" << endl;
}

int main(){
  show(8);
  show(string("abc"));
}
\end{lstlisting}

\subsubsection{Variable Dict}

Use variable dict template to write print function

\begin{lstlisting}
#include<iostream>
#include<string>
using namespace std;

template<typename T> void print(const T&t){
  cout << t << endl;
}

template<typename T, typename... Y> void print(const T& first, Y... y){
  cout << first << " ";
  print(y...);
}

int main(){
  print(1, 2, 3, "abc");
}
\end{lstlisting}

\subsubsection{Perfect Forwarding}
\subsubsection{Smart Pointer}

\subsection{Python}
\subsubsection{Dataframe Data-structure}

\subsection{Java}

\subsection{UNIX Bash}

\subsection{MongoDB}